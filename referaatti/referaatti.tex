% --- Template for thesis / report with tktltiki2 class ---
% 
% last updated 2013/02/15 for tkltiki2 v1.02

\documentclass[finnish]{tktltiki2}

% tktltiki2 automatically loads babel, so you can simply
% give the language parameter (e.g. finnish, swedish, english, british) as
% a parameter for the class: \documentclass[finnish]{tktltiki2}.
% The information on title and abstract is generated automatically depending on
% the language, see below if you need to change any of these manually.
% 
% Class options:
% - grading                 -- Print labels for grading information on the front page.
% - disablelastpagecounter  -- Disables the automatic generation of page number information
%                              in the abstract. See also \numberofpagesinformation{} command below.
%
% The class also respects the following options of article class:
%   10pt, 11pt, 12pt, final, draft, oneside, twoside,
%   openright, openany, onecolumn, twocolumn, leqno, fleqn
%
% The default font size is 11pt. The paper size used is A4, other sizes are not supported.
%
% rubber: module pdftex

% --- General packages ---

\usepackage[utf8]{inputenc}
\usepackage[T1]{fontenc}
\usepackage{lmodern}
\usepackage{microtype}
\usepackage{amsfonts,amsmath,amssymb,amsthm,booktabs,color,enumitem,graphicx}
\usepackage[pdftex,hidelinks]{hyperref}

% Automatically set the PDF metadata fields
\makeatletter
\AtBeginDocument{\hypersetup{pdftitle = {\@title}, pdfauthor = {\@author}}}
\makeatother

% References ilman bracket

\makeatletter 
\renewcommand\@biblabel[1]{#1} 
\makeatother

% --- Language-related settings ---
%
% these should be modified according to your language

% babelbib for non-english bibliography using bibtex
\usepackage[fixlanguage]{babelbib}
\selectbiblanguage{finnish}

% add bibliography to the table of contents
\usepackage[nottoc]{tocbibind}
% tocbibind renames the bibliography, use the following to change it back
\settocbibname{Lähteet}

% --- Theorem environment definitions ---

\newtheorem{lau}{Lause}
\newtheorem{lem}[lau]{Lemma}
\newtheorem{kor}[lau]{Korollaari}

\theoremstyle{definition}
\newtheorem{maar}[lau]{Määritelmä}
\newtheorem{ong}{Ongelma}
\newtheorem{alg}[lau]{Algoritmi}
\newtheorem{esim}[lau]{Esimerkki}

\theoremstyle{remark}
\newtheorem*{huom}{Huomautus}


% --- tktltiki2 options ---
%
% The following commands define the information used to generate title and
% abstract pages. The following entries should be always specified:

\title{Mutaatiotestaus oliojärjestelmissä}
\author{Eveliina Pakarinen}
\date{\today}
\level{Referaatti}
\abstract{Referaatin tiivistelmä}

% The following can be used to specify keywords and classification of the paper:

\keywords{mutaatiotestaus, oliojärjestelmät, Java}

% classification according to ACM Computing Classification System (http://www.acm.org/about/class/)
% This is probably mostly relevant for computer scientists
% uncomment the following; contents of \classification will be printed under the abstract with a title
% "ACM Computing Classification System (CCS):"
% \classification{}

% If the automatic page number counting is not working as desired in your case,
% uncomment the following to manually set the number of pages displayed in the abstract page:
%
% \numberofpagesinformation{16 sivua + 10 sivua liitteissä}
%
% If you are not a computer scientist, you will want to uncomment the following by hand and specify
% your department, faculty and subject by hand:
%
% \faculty{Matemaattis-luonnontieteellinen}
% \department{Tietojenkäsittelytieteen laitos}
% \subject{Tietojenkäsittelytiede}
%
% If you are not from the University of Helsinki, then you will most likely want to set these also:
%
% \university{Helsingin Yliopisto}
% \universitylong{HELSINGIN YLIOPISTO --- HELSINGFORS UNIVERSITET --- UNIVERSITY OF HELSINKI} % displayed on the top of the abstract page
% \city{Helsinki}
%


\begin{document}

% --- Front matter ---

\frontmatter      % roman page numbering for front matter

\maketitle        % title page
%\makeabstract     % abstract page

%\tableofcontents  % table of contents

% --- Main matter ---

\mainmatter       % clear page, start arabic page numbering

%\section{Mutaatiotestaus oliojärjestelmissä}

% Write some science here.

% Tähän tekstiä. \\ Esimerkki lähdeviitteestä: ~\cite{toinen}.

Olioperustaisen ohjelmoinnin kehityksen myötä klassisia ohjelmistojen testausmenetelmiä on jouduttu sopeuttamaan uusiin vaatimuksiin, joita oliojärjestelmien kattava ja laadukas testaaminen vaatii. Vaikka olioperustainen ohjelmoiti ratkaisee joitakin proseduraalisen ohjelmoinnin suunnittelu- ja toteutusongelmia, se tuo mukanaan myös uusia haasteita, jotka vaativat uusien testaus- ja analysointimenetelmien kehittämistä~\cite{Mariani:Pezze:2008}. Näitä olio-ohjelmoinnin erityispiirteitä ovat muun muuassa kapselointi, perintä, dynaaminen linkitys ja polymorfismi~\cite{jokulähde}. 

Testausta on käytetty ohjelmistokehityksessä ohjelmiston laadun varmistamiseen ja samalla auttamaan virheiden havaitsemisessa jo kehitysvaiheen aikana. Ohjelmistojen testaamisen ensisijainen tavoite on siis paljastaa virheitä, joiden havaitseminen muiden laadunvarmistusmenetelmien avulla olisi työlästä tai mahdotonta. Lisäksi testauksen avulla on pyritty varmistamaan, että ohjelma toimii sille asetettujen vaatimusten mukaisesti~\cite[s. 59]{Binder:1999}. 

Testaukseen liittyy kuitenkin myös absoluuttisia rajoituksia. Sen avulla ei voi esimerkiksi määrittää ohjelmiston oikeellisuutta. Tähän liittyen Dijkstra kirjoitti seuraavan korollaarin liittyen ohjelmistojen oikeaksi todistamiseen: "Ohjelmiston testausta voidaan käyttää näyttämään virheiden olemassaolo, mutta ei koskan niiden puuttumista~\cite[s. 6]{Dahl:Dijkstra:Hoare:1972}." 

Ohjelmistoja voidaan testata monella tasolla ja testien suunnittelussa voidaan käyttää erilaisia strategioita. Testauksen tasot koostuvat joukoista ohjelman komponentteja, joita halutaan testata. Alimmalla testauksen tasolla testattavat ohjelman osat ovat pienimpiä mahdollisia suoritettavissaolevia komponentteja. Tämän tason testausta kutsutaan \textit{yksikkötestaukseksi} (\textit{unit testing}) ja testattavat osat ovat olio-ohjelmissa esimerkiksi yksittäisiä metodeja tai luokkia~\cite[s. 45]{Binder:1999}. Tällä testauksen tasolla testien suunnittelussa voidaan käyttää ohjelman sisäiseen rakenteeseen perustuvaa suunnittelumallia, jota kutsutaan \textit{lasilaatikkotestaukseksi} (\textit{white box testing}). Tässä mallissa ohjelmiston lähdekoodia käytetään apuna testien valmistamisessa~\cite[s. 52]{Binder:1999}.

Yksikkötestauksesta seuraava taso ylöspäin on \textit{integraatiotestaus} (\textit{integration testing}), jossa tarkastellaan jonkin järjestelmän tai sen osien yhteistoimintaa ja keskinäistä kommunikointia testaamalla osien välisiä rajapintoja. Olio-ohjelmissa luokkien muodostuminen perinnän avulla ja niiden koostuminen toisten luokkien olioista aiheuttaa sen, että integraatiotestaukselle on tarvetta jo aikaisin olioperustaisen ohjelmoinnin alkuvaiheessa~\cite[s. 45]{Binder:1999}.

Jo valmista integroitua sovellusta testataan \textit{järjestelmätestauksen} (\textit{system testing}) avulla. Tällä testauksen tasolla keskitytään vain valmiissa sovelluksessa esiintyvien piirteiden testaamiseen. Testauksen kohteena voi olla esimerkiksi sovelluksen toiminnallisuus, suorituskyky tai sen kestämän kuormituksen määrä~\cite[s. 45]{Binder:1999}. Tämän tason testien suunnittelustrategiana voidaan käyttää ohjelmiston ulkoisen toiminnallisuuden analysointia. Tällaista testausstrategiaa kutsutaan \textit{mustalaatikkotestaukseksi} (\textit{black box testing}) tai \textit{funktionaaliseksi testaukseksi} (\textit{functional testing})~\cite[s. 52]{Binder:1999}. Kuitenkin sekä lasi- että mustalaatikkotestausta voidaan käyttää kaikilla testauksen tasoilla testien suunnittelun apuna. 

Vaikka testauksen avulla ei voikaan varmistua ohjelmiston oikeellisuudesta, voidaan sitä käyttää välineenä ohjelmiston laadun parantamisessa. Kuitenkin testaukseen liittyy myös rajoitus ja epävarmuus käytettävän testausjärjestelmän oikeellisuudesta ja oikeellisuuden varmistamisesta~\cite{Manna:Waldinger:1978}. Tämä herättää kysymyksen siitä, kuka voi valvoa valvojia eli kuinka valvotaan ohjelmiston testien laatua~\cite{JOTAKINTÄHÄN}.

Yksi strategia testien laadun tarkkailuun on \textit{mutaatiotestaus} (\textit{mutation testing})~\cite{tähänalkuperäinenlähdemutaatiotestaukselle}. Mutaatiotestauksessa ohjelmiston alkuperäistä lähdekoodia muutetaan \textit{mutaatio-operaattorien} (\textit{mutation operators}) avulla lisäämällä sen syntaksiin virheitä. Näitä virheellisiä ohjelmakoodin versioita kutsutaan \textit{mutanteiksi} (\textit{mutants})~\cite{Ma:Harrold:Kwon:2006}. Mutanttien generoinnin jälkeen ohjelmiston alkuperäiset testit suoritetaan jokaisen mutantin kohdalla, ja toivotaan, että testit eivät mene läpi. Jos testi tuottaa väärän lopputuloksen, se tarkoittaa, että lähdekoodiin tehdyt muutokset on havaittu ja mutantti on tapettu. 

Mutaatotestausprosessi tuottaa lopputuloksena \textit{mutaatiopistemäärän} (\textit{mutation adequacy score}), jonka avulla ohjelmiston testien laadukkuutta ja kykyä havaita lähdekoodissa olevia vikoja voidaan arvioida. Mutaatiotestaus on virheperustainen testausmenetelmä, jonka taustaperiaatteena on ohjelmoijien tekemien ohjelmointivirheiden simulointi~\cite{Jia:Harman:2011}. Tavoitteena mutaatiotestauksessa on tutkia, ovatko ohjelmistoa varten tehdyt testit laadukkaita ja havaitsevatko ne kattavasti ohjelmistossa mahdollisesti esiintyvät virheet ja ongelmat. 


% --- References ---
%
% bibtex is used to generate the bibliography. The babplain style
% will generate numeric references (e.g. [1]) appropriate for theoretical
% computer science. If you need alphanumeric references (e.g [Tur90]), use
%
\newpage
\bibliographystyle{babalpha-lf}
%
% instead.

%\bibliographystyle{babplain-lf}
\bibliography{references-fi}


% --- Appendices ---

% uncomment the following

% \newpage
% \appendix
% 
% \section{Esimerkkiliite}

\end{document}
