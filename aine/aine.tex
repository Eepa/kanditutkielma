% --- Template for thesis / report with tktltiki2 class ---
% 
% last updated 2013/02/15 for tkltiki2 v1.02

\documentclass[finnish, grading]{tktltiki2}

% tktltiki2 automatically loads babel, so you can simply
% give the language parameter (e.g. finnish, swedish, english, british) as
% a parameter for the class: \documentclass[finnish]{tktltiki2}.
% The information on title and abstract is generated automatically depending on
% the language, see below if you need to change any of these manually.
% 
% Class options:
% - grading                 -- Print labels for grading information on the front page.
% - disablelastpagecounter  -- Disables the automatic generation of page number information
%                              in the abstract. See also \numberofpagesinformation{} command below.
%
% The class also respects the following options of article class:
%   10pt, 11pt, 12pt, final, draft, oneside, twoside,
%   openright, openany, onecolumn, twocolumn, leqno, fleqn
%
% The default font size is 11pt. The paper size used is A4, other sizes are not supported.
%
% rubber: module pdftex

% --- General packages ---

\usepackage[utf8]{inputenc}
\usepackage[T1]{fontenc}
\usepackage{lmodern}
\usepackage{microtype}
\usepackage{amsfonts,amsmath,amssymb,amsthm,booktabs,color,enumitem,graphicx}
\usepackage[pdftex,hidelinks]{hyperref}

% Automatically set the PDF metadata fields
\makeatletter
\AtBeginDocument{\hypersetup{pdftitle = {\@title}, pdfauthor = {\@author}}}
\makeatother

% References ilman bracket

\makeatletter 
\renewcommand\@biblabel[1]{#1} 
\makeatother

% --- Language-related settings ---
%
% these should be modified according to your language

% babelbib for non-english bibliography using bibtex
\usepackage[fixlanguage]{babelbib}
\selectbiblanguage{finnish}

% add bibliography to the table of contents
\usepackage[nottoc]{tocbibind}
% tocbibind renames the bibliography, use the following to change it back
\settocbibname{Lähteet}

% --- Theorem environment definitions ---

\newtheorem{lau}{Lause}
\newtheorem{lem}[lau]{Lemma}
\newtheorem{kor}[lau]{Korollaari}

\theoremstyle{definition}
\newtheorem{maar}[lau]{Määritelmä}
\newtheorem{ong}{Ongelma}
\newtheorem{alg}[lau]{Algoritmi}
\newtheorem{esim}[lau]{Esimerkki}

\theoremstyle{remark}
\newtheorem*{huom}{Huomautus}


% --- tktltiki2 options ---
%
% The following commands define the information used to generate title and
% abstract pages. The following entries should be always specified:

\title{Mutaatiotestaus oliojärjestelmissä}
\author{Eveliina Pakarinen}
\date{\today}
\level{Aine}
\abstract{Aineen tiivistelmä}

% The following can be used to specify keywords and classification of the paper:

\keywords{mutaatiotestaus, oliojärjestelmät, Java}

% classification according to ACM Computing Classification System (http://www.acm.org/about/class/)
% This is probably mostly relevant for computer scientists
% uncomment the following; contents of \classification will be printed under the abstract with a title
% "ACM Computing Classification System (CCS):"
% \classification{}

% If the automatic page number counting is not working as desired in your case,
% uncomment the following to manually set the number of pages displayed in the abstract page:
%
% \numberofpagesinformation{16 sivua + 10 sivua liitteissä}
%
% If you are not a computer scientist, you will want to uncomment the following by hand and specify
% your department, faculty and subject by hand:
%
% \faculty{Matemaattis-luonnontieteellinen}
% \department{Tietojenkäsittelytieteen laitos}
% \subject{Tietojenkäsittelytiede}
%
% If you are not from the University of Helsinki, then you will most likely want to set these also:
%
% \university{Helsingin Yliopisto}
% \universitylong{HELSINGIN YLIOPISTO --- HELSINGFORS UNIVERSITET --- UNIVERSITY OF HELSINKI} % displayed on the top of the abstract page
% \city{Helsinki}
%


\begin{document}

% --- Front matter ---

\frontmatter      % roman page numbering for front matter

\maketitle        % title page
\makeabstract     % abstract page

\tableofcontents  % table of contents

% --- Main matter ---

\mainmatter       % clear page, start arabic page numbering

% Write some science here.

% Tähän tekstiä. \\ Esimerkki lähdeviitteestä: ~\cite{toinen}.

%\[\frac{a}{b}\]

\section{Johdanto [(1s)]}

\textbf{Tähän johdantoa}

%Ohjelmistoja voidaan testata monella tasolla. Testauksen tasot muodostetaan määrittämällä joukko ohjelman osia eli komponentteja, joita halutaan testata. Alimmalla testauksen tasolla testattavat ohjelman osat ovat pienimpiä mahdollisia suoritettavissa olevia komponentteja. Tämän tason testausta kutsutaan \textit{yksikkötestaukseksi} (\textit{unit testing}) ja testattavat osat ovat olio-ohjelmissa esimerkiksi yksittäisiä metodeja tai luokkia~\cite[s. 45]{Binder:1999}.
%
%Yksikkötestauksesta seuraava taso ylöspäin on \textit{integraatiotestaus} (\textit{integration testing}), jossa tarkastellaan järjestelmän tai sen osien yhteistoimintaa ja keskinäistä kommunikointia testaamalla osien välisiä rajapintoja. Olio-ohjelmissa luokkien muodostuminen perinnän avulla ja luokkien koostuminen toisten luokkien olioista aiheuttaa sen, että integraatiotestaukselle on tarvetta jo olioperustaisen ohjelmoinnin alkuvaiheessa~\cite[s. 45]{Binder:1999}.
%
%Valmista integroitua sovellusta testataan \textit{järjestelmätestauksen} (\textit{system testing}) avulla. Tällä testauksen tasolla keskitytään vain valmiissa sovelluksessa esiintyvien piirteiden testaamiseen. Testauksen kohteena voi olla esimerkiksi sovelluksen toiminnallisuus, suorituskyky tai sovelluksen kestämä kuormitus~\cite[s. 45]{Binder:1999}.
%
%Testien suunnittelussa ja kehittämisessä voidaan myös käyttää erilaisia suunnittelumalleja, joiden avulla kuvataan testien suunnitteluun käytettävää näkökulmaa. Ohjelman sisäiseen rakenteeseen eli lähdekoodin tuntemukseen perustuvaa testien suunnittelumallia kutsutaan \textit{white box -testaukseksi}. \textit{Black box -testaukseksi} tai \textit{funktionaaliseksi testaukseksi} kutsutussa suunnittelumallissa testejä suunnitellaan analysoimalla ohjelmiston ulkoista toiminnallisuutta~\cite[s. 52]{Binder:1999}.

%Mutanttien generoinnin jälkeen ohjelmiston alkuperäiset testit suoritetaan jokaisen mutantin kohdalla. Tavoitteena on, että testien avulla havaitaan lähdekoodiin tehdyt muutokset. Jos alkuperäiset testit eivät mene läpi, se tarkoittaa, että mutantti on tapettu eli lähdekoodiin tehdyt muutokset on havaittu~\cite[s. 9]{Kim:Clark:McDermid:2000}. 
%
%Mutaatiotestausprosessi tuottaa lopputuloksena \textit{mutaatiopistemäärän} (\textit{mutation adequacy score}), jonka avulla voidaan arvioida ohjelmiston testien laadukkuutta ja kykyä havaita lähdekoodissa olevia vikoja. 



Kuvaus aineesta ja sen osioista?


\section{Testaus oliojärjestelmissä (5s) [2s]}

\textbf{Tähän tietoa siitä, mikä on oliojärjestelmä. Aiheen esittely.}

Olioperustaisen ohjelmoinnin kehityksen myötä klassisia ohjelmistojen testausmenetelmiä on sopeutettu mahdollistamaan \textit{oliojärjestelmien} (\textit{object oriented systems}) kattava ja laadukas testaaminen. Vaikka olioperustainen ohjelmointi ratkaisee joitakin proseduraalisen ohjelmoinnin suunnittelu- ja toteutusongelmia, olio-ohjelmoinnin mukana tulevat uudet haasteet vaativat uusien testaus- ja analysointimenetelmien kehittämistä. Olio-ohjelmoinnissa testauksen tekevät haastavaksi muun muuassa kapselointi, perintä, dynaaminen sidonta ja polymorfismi~\cite[s. 86]{Mariani:Pezze:2008}.

Testausta käytetään ohjelmistokehityksessä ohjelmiston laadun varmistamiseen ja auttamaan virheiden havaitsemisessa jo kehitysvaiheen aikana. Ohjelmistojen testaamisen ensisijainen tavoite on paljastaa virheitä, joiden havaitseminen muiden laadunvarmistusmenetelmien avulla olisi työlästä tai mahdotonta. Lisäksi testauksen avulla on pyritty varmistamaan, että ohjelma toimii sille asetettujen vaatimusten mukaisesti~\cite[s. 59]{Binder:1999} (lähde pitää siirtää johonkin toiseen kohtaan/lisätä jokaisen lauseen perään lähde). 

\subsection{Testauksen erityispiirteet oliojärjestelmissä/yleisesittely (onko tarpeellinen omana osionaan?)}

\textbf{Onko tämä tarpeellinen oma osio vai lisätäänkö tiedot luvun introon?} 

 Näitä olio-ohjelmoinnin erityispiirteitä ovat muun muuassa kapselointi, perintä, \textit{dynaaminen linkitys} (\textit{dynamic binding}) ja polymorfismi~\cite[s. 86]{Mariani:Pezze:2008}.


\subsection{Testauksen tasot}

\textbf{Avataanko testauksen tasoja enemmän? Kuinka paljon niistä on kerrottava?}

Ohjelmistoja voidaan testata monella tasolla. Testauksen tasot muodostetaan määrittämällä joukko ohjelman osia eli komponentteja, joita halutaan testata. Alimmalla testauksen tasolla testattavat ohjelman osat ovat pienimpiä mahdollisia suoritettavissa olevia komponentteja. Tämän tason testausta kutsutaan \textit{yksikkötestaukseksi} (\textit{unit testing}) ja testattavat osat ovat olio-ohjelmissa esimerkiksi yksittäisiä metodeja tai luokkia~\cite[s. 45]{Binder:1999}.

Yksikkötestauksesta seuraava taso ylöspäin on \textit{integraatiotestaus} (\textit{integration testing}), jossa tarkastellaan järjestelmän tai sen osien yhteistoimintaa ja keskinäistä kommunikointia testaamalla osien välisiä rajapintoja. Olio-ohjelmissa luokkien muodostuminen perinnän avulla ja luokkien koostuminen toisten luokkien olioista aiheuttaa, että integraatiotestaukselle on tarvetta olioperustaisessa ohjelmoinnissa jo ohjelmoinnin alkuvaiheessa~\cite[s. 45]{Binder:1999}.

Valmista integroitua sovellusta testataan \textit{järjestelmätestauksen} (\textit{system testing}) avulla. Tällä testauksen tasolla keskitytään vain valmiissa sovelluksessa esiintyvien piirteiden testaamiseen. Testauksen kohteena voi olla esimerkiksi sovelluksen toiminnallisuus, suorituskyky tai sovelluksen kestämä kuormitus~\cite[s. 45]{Binder:1999}.


\subsection{Testien suunnittelumallit}

\textbf{Kuinka paljon näitä avataan lisää? Miten paljon on relevanttia tutkielman kannalta?}

Testien suunnittelussa ja kehittämisessä voidaan myös käyttää erilaisia suunnittelumalleja, joiden avulla kuvataan testien suunnitteluun käytettävää näkökulmaa. Ohjelman sisäisen rakenteen eli lähdekoodin tuntemukseen perustuvaa testien suunnittelumallia kutsutaan \textit{white box -testaukseksi}. \textit{Black box -testaukseksi} tai \textit{funktionaaliseksi testaukseksi} kutsutussa suunnittelumallissa testejä suunnitellaan analysoimalla ohjelmiston ulkoista toiminnallisuutta~\cite[s. 52]{Binder:1999}.


\subsection{Testauksen rajoitukset}

\textbf{Kuinka paljon rajoituksista on kerrottava? Niitä on aika paljon erilaisia. Mitkä niistä ovat relevantteja tämän tutkielman kannalta?}

Testaukseen kuuluu kuitenkin myös (tiettyjä) rajoituksia. Sen avulla ei voi esimerkiksi todeta ohjelmiston oikeellisuutta. Ohjelmistojen oikeaksi todistamiseen ja testauksen rajoituksiin liittyen Dijkstra kirjoitti/on kirjoittanut seuraavan korollaarin: \textit{''Program testing can be used to show the presence of bugs, but never to show their absence!''}~\cite[s. 6]{Dahl:Dijkstra:Hoare:1972}. 

\section{Mutaatiotestaus oliojärjestelmissä (5-7s) [3s]}

\textbf{Haasteisiin voidaan vastata mutaatiotestauksen avulla}

Vaikka testauksen avulla ei voikaan varmistua ohjelmiston oikeellisuudesta, voidaan testausta käyttää välineenä ohjelmiston laadun parantamisessa. Testaukseen sisältyvien rajoitusten lisäksi siihen liittyy myös epävarmuutta käytettävän testausjärjestelmän oikeellisuudesta ja oikeellisuuden varmistamisesta~\cite[s. 209]{Manna:Waldinger:1978}. Tämä herättää kysymyksen siitä, kuka voi valvoa valvojia eli kuinka varmistetaan ohjelmiston testien laadukkuus.

Yksi strategia testien laadun varmistamiseen on \textit{mutaatiotestaus}. Mutaatiotestauksessa ohjelmiston alkuperäistä lähdekoodia käsitellään \textit{mutaatio-operaattoreilla} (\textit{mutation operators}), jotka muuntavat koodia muodostaen siitä virheellisiä versioita. Näitä virheellisiä ohjelmakoodin versioita kutsutaan \textit{mutanteiksi}~\cite[s. 869]{Ma:Harrold:Kwon:2006}. 

\subsection{Mutaatiotestauksen yleisesittely}

\textbf{Laajentaa sopivista kohdista yleisesittelyä.}

Yksi strategia testien laadun varmistamiseen on \textit{mutaatiotestaus}. Mutaatiotestauksessa ohjelmiston alkuperäistä lähdekoodia käsitellään \textit{mutaatio-operaattoreilla} (\textit{mutation operators}), jotka muuntavat koodia muodostaen siitä virheellisiä versioita. Näitä virheellisiä ohjelmakoodin versioita kutsutaan \textit{mutanteiksi}~\cite[s. 869]{Ma:Harrold:Kwon:2006}. 

Mutanttien generoinnin jälkeen ohjelmiston alkuperäiset testit suoritetaan jokaisen mutantin kohdalla. Tavoitteena on, että testien avulla havaitaan lähdekoodiin tehdyt muutokset. Jos alkuperäiset testit eivät mene läpi, se tarkoittaa, että mutantti on tapettu eli lähdekoodiin tehdyt muutokset on havaittu~\cite[s. 9]{Kim:Clark:McDermid:2000}. 

Mutaatiotestausprosessi tuottaa lopputuloksena \textit{mutaatiopistemäärän} (\textit{mutation adequacy score}), jonka avulla voidaan arvioida ohjelmiston testien laadukkuutta ja kykyä havaita lähdekoodissa olevia vikoja. 

Mutaatiotestaus on \textit{virheperustainen testausmenetelmä} (\textit{fault based jotakin}), jonka periaatteena on ohjelmoijien tekemien ohjelmointivirheiden simulointi~\cite[s. 649]{Jia:Harman:2011}. Tavoitteena mutaatiotestauksessa on tutkia, ovatko ohjelmistoa varten tehdyt testit laadukkaita ja havaitaanko niillä kattavasti ohjelmistossa mahdollisesti esiintyvät virheet ja ongelmat. 


\subsection{Mutaatiotestauksen piirteet oliojärjestelmissä (erot muihin järjestelmiin?)}

\textbf{Erityisesti juuri nuo mutaatio-operaattorit ja minkälaisia erilaisia niitä on. Esittelyä tulee mutaatio-operaattoreista.}

\section{Mutaatiotestauksen haasteet (5-7s) [3s]}

\textbf{Misi mutaatiotestaus ei ole päätynyt suureen suosioon/käyttöön? Valitaan muutama haaste ja esitellään niitä? Kerrotaan että myös muita haasteita, mutta ei esitellä niitä?}

\subsection{Ekvivalentit mutantit}

\textbf{Ratkaisematon ongelma. Pitäisi avata ja kehittää esimerkki.}

\subsection{Tehokkuusongelmat}

\textbf{Laitteisto, testien määrä, ihmisten aika tulee vastaan. Esimerkkejä.}

\subsection{Muita ongelmia?}

\textbf{Varmasti on muita, mutta kuinka paljon niitä mahtuu tähän esitelmään?}

\subsection{Käytännöntoteutus (miten käytännössä on yritetty kiertää haasteet)}

\textbf{Onko tämä relevantti aihealue käsitellä?}

%\section{Mutaatiotestauksen tulevaisuus (2s)}

%\textbf{Osaanko antaa katsausta tulevaisuuteen? Oma arvio sen perusteella mitä on lukenut? Onko se tämän kandin tarkoitus? Mahdollisia käsiteltäviä alueita väliotsikoissa (ovatko väliotsikot edes tarpeen?).}

%\subsection{Ennusteita tulevalle käytölle (teollisuudessa)}

%\subsection{Mahdolliset kehityssuunnat (tutkimuksessa)}

%\section{Miten saa evaluointia/analyysiä? Mihin lukuihin se sopii?}

%\textbf{Mihin kohtiin evaluointi ja analyysi sopivat, jotta tästä ei tule mutaatiotestauksen/testauksen oppikirjaa? Miten sitä tehdään? Vertailemalla, ennustamalla tulevaa, jotakin muuta?}

\section{Yhteenveto [(1s)]}

\textbf{Conclusion.}

 






 


%Tällä (yksikkö)testauksen tasolla testien suunnittelussa voidaan käyttää ohjelman sisäiseen rakenteeseen perustuvaa suunnittelumallia, jota kutsutaan \textit{white box -testaukseksi} (\textit{white box testing}). Tässä mallissa ohjelmiston lähdekoodia käytetään apuna testien valmistamisessa~\cite[s. 52]{Binder:1999}.

%Tämän tason testien suunnittelustrategiana voidaan käyttää ohjelmiston ulkoisen toiminnallisuuden analysointia. Tällaista testausstrategiaa kutsutaan \textit{black box -testaukseksi} (\textit{black box testing}) tai \textit{funktionaaliseksi testaukseksi} (\textit{functional testing})~\cite[s. 52]{Binder:1999}. Kuitenkin sekä white box - että black box -testausta voidaan käyttää kaikilla testauksen tasoilla testien suunnittelun apuna. 


% --- References ---
%
% bibtex is used to generate the bibliography. The babplain style
% will generate numeric references (e.g. [1]) appropriate for theoretical
% computer science. If you need alphanumeric references (e.g [Tur90]), use
%
\newpage
\bibliographystyle{babalpha-lf}
%
% instead.

%\bibliographystyle{babplain-lf}
\bibliography{references-fi}


% --- Appendices ---

% uncomment the following

% \newpage
% \appendix
% 
% \section{Esimerkkiliite}

\end{document}
