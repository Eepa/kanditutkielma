% --- Template for thesis / report with tktltiki2 class ---
% 
% last updated 2013/02/15 for tkltiki2 v1.02

\documentclass[finnish, grading]{tktltiki2}

% tktltiki2 automatically loads babel, so you can simply
% give the language parameter (e.g. finnish, swedish, english, british) as
% a parameter for the class: \documentclass[finnish]{tktltiki2}.
% The information on title and abstract is generated automatically depending on
% the language, see below if you need to change any of these manually.
% 
% Class options:
% - grading                 -- Print labels for grading information on the front page.
% - disablelastpagecounter  -- Disables the automatic generation of page number information
%                              in the abstract. See also \numberofpagesinformation{} command below.
%
% The class also respects the following options of article class:
%   10pt, 11pt, 12pt, final, draft, oneside, twoside,
%   openright, openany, onecolumn, twocolumn, leqno, fleqn
%
% The default font size is 11pt. The paper size used is A4, other sizes are not supported.
%
% rubber: module pdftex

% --- General packages ---

\usepackage[utf8]{inputenc}
\usepackage[T1]{fontenc}
\usepackage{lmodern}
\usepackage{microtype}
\usepackage{amsfonts,amsmath,amssymb,amsthm,booktabs,color,enumitem,graphicx}
\usepackage[pdftex,hidelinks]{hyperref}

% Automatically set the PDF metadata fields
\makeatletter
\AtBeginDocument{\hypersetup{pdftitle = {\@title}, pdfauthor = {\@author}}}
\makeatother

% References ilman bracket

\makeatletter 
\renewcommand\@biblabel[1]{#1} 
\makeatother

% --- Language-related settings ---
%
% these should be modified according to your language

% babelbib for non-english bibliography using bibtex
\usepackage[fixlanguage]{babelbib}
\selectbiblanguage{finnish}

% add bibliography to the table of contents
\usepackage[nottoc]{tocbibind}
% tocbibind renames the bibliography, use the following to change it back
\settocbibname{Lähteet}

% --- Theorem environment definitions ---

\newtheorem{lau}{Lause}
\newtheorem{lem}[lau]{Lemma}
\newtheorem{kor}[lau]{Korollaari}

\theoremstyle{definition}
\newtheorem{maar}[lau]{Määritelmä}
\newtheorem{ong}{Ongelma}
\newtheorem{alg}[lau]{Algoritmi}
\newtheorem{esim}[lau]{Esimerkki}

\theoremstyle{remark}
\newtheorem*{huom}{Huomautus}


% --- tktltiki2 options ---
%
% The following commands define the information used to generate title and
% abstract pages. The following entries should be always specified:

\title{Mutaatiotestaus oliojärjestelmissä}
\author{Eveliina Pakarinen}
\date{\today}
\level{Aine}
\abstract{Aineen tiivistelmä}

% The following can be used to specify keywords and classification of the paper:

\keywords{mutaatiotestaus, oliojärjestelmät, Java}

% classification according to ACM Computing Classification System (http://www.acm.org/about/class/)
% This is probably mostly relevant for computer scientists
% uncomment the following; contents of \classification will be printed under the abstract with a title
% "ACM Computing Classification System (CCS):"
% \classification{}

% If the automatic page number counting is not working as desired in your case,
% uncomment the following to manually set the number of pages displayed in the abstract page:
%
% \numberofpagesinformation{16 sivua + 10 sivua liitteissä}
%
% If you are not a computer scientist, you will want to uncomment the following by hand and specify
% your department, faculty and subject by hand:
%
% \faculty{Matemaattis-luonnontieteellinen}
% \department{Tietojenkäsittelytieteen laitos}
% \subject{Tietojenkäsittelytiede}
%
% If you are not from the University of Helsinki, then you will most likely want to set these also:
%
% \university{Helsingin Yliopisto}
% \universitylong{HELSINGIN YLIOPISTO --- HELSINGFORS UNIVERSITET --- UNIVERSITY OF HELSINKI} % displayed on the top of the abstract page
% \city{Helsinki}
%


\begin{document}

% --- Front matter ---

\frontmatter      % roman page numbering for front matter

\maketitle        % title page
\makeabstract     % abstract page

\tableofcontents  % table of contents

% --- Main matter ---

\mainmatter       % clear page, start arabic page numbering

% Write some science here.

% Tähän tekstiä. \\ Esimerkki lähdeviitteestä: ~\cite{toinen}.

%\[\frac{a}{b}\]

\section{Johdanto [1s]}

%\textcolor{cyan}{Tutkimuskysymys: On tutkittu, että mutaatiotestauksesta on hyötyä testien laadun parantamisessa ja testikattavuuden tarkastelussa/kasvattamisessa. Miksi mutaatiotestausta ei kuitenkaan käytetä laajasti näiden asioiden tekemiseen?}
%
%\textcolor{red}{Vastaus: Koska mutaatiotestaus on tällä hetkellä vielä työlästä ja tehotonta sen tuomaan hyötyyn nähden. Koska mutaatiotestauksessa on ratkaisemattomia ongelmia (esim. ekvivalentit mutantit), joita ei voi automatisoida --> vaativat ihmisten panostusta --> lisää työläyttä.} 
%
%\textcolor{blue}{Tai toisinpäin: Miksi mutaatiotestausta käytettäisiin/pitäisi käyttää enemmän testaamiseen, vaikka mutaatiotestaus on tehotonta/työlästä?}
%
%\textcolor{red}{Vastaus: Koska mutaatiotestausta voidaan käyttää testien laadun parantamiseen tai testikattavuuden kasvattamiseen. Koska mutaatiotestauksen avulla voi varmistaa, että testien laatu on hyvä ja näin ollen parantaa samalla ohjelmiston laatua --> ohjelmistosta saadaan suurin osa ongelmista ja virheistä pois testien avulla. Koska mutaatiotestauksella voi paikata normaaliin testaukseen jääneitä puutteita.} 

%Ohjelmistoja voidaan testata monella tasolla. Testauksen tasot muodostetaan määrittämällä joukko ohjelman osia eli komponentteja, joita halutaan testata. Alimmalla testauksen tasolla testattavat ohjelman osat ovat pienimpiä mahdollisia suoritettavissa olevia komponentteja. Tämän tason testausta kutsutaan \textit{yksikkötestaukseksi} (\textit{unit testing}) ja testattavat osat ovat olio-ohjelmissa esimerkiksi yksittäisiä metodeja tai luokkia~\cite[s. 45]{Binder:1999}.
%
%Yksikkötestauksesta seuraava taso ylöspäin on \textit{integraatiotestaus} (\textit{integration testing}), jossa tarkastellaan järjestelmän tai sen osien yhteistoimintaa ja keskinäistä kommunikointia testaamalla osien välisiä rajapintoja. Olio-ohjelmissa luokkien muodostuminen perinnän avulla ja luokkien koostuminen toisten luokkien olioista aiheuttaa sen, että integraatiotestaukselle on tarvetta jo olioperustaisen ohjelmoinnin alkuvaiheessa~\cite[s. 45]{Binder:1999}.
%
%Valmista integroitua sovellusta testataan \textit{järjestelmätestauksen} (\textit{system testing}) avulla. Tällä testauksen tasolla keskitytään vain valmiissa sovelluksessa esiintyvien piirteiden testaamiseen. Testauksen kohteena voi olla esimerkiksi sovelluksen toiminnallisuus, suorituskyky tai sovelluksen kestämä kuormitus~\cite[s. 45]{Binder:1999}.
%
%Testien suunnittelussa ja kehittämisessä voidaan myös käyttää erilaisia suunnittelumalleja, joiden avulla kuvataan testien suunnitteluun käytettävää näkökulmaa. Ohjelman sisäiseen rakenteeseen eli lähdekoodin tuntemukseen perustuvaa testien suunnittelumallia kutsutaan \textit{white box -testaukseksi}. \textit{Black box -testaukseksi} tai \textit{funktionaaliseksi testaukseksi} kutsutussa suunnittelumallissa testejä suunnitellaan analysoimalla ohjelmiston ulkoista toiminnallisuutta~\cite[s. 52]{Binder:1999}.

%Mutanttien generoinnin jälkeen ohjelmiston alkuperäiset testit suoritetaan jokaisen mutantin kohdalla. Tavoitteena on, että testien avulla havaitaan lähdekoodiin tehdyt muutokset. Jos alkuperäiset testit eivät mene läpi, se tarkoittaa, että mutantti on tapettu eli lähdekoodiin tehdyt muutokset on havaittu~\cite[s. 9]{Kim:Clark:McDermid:2000}. 
%
%Mutaatiotestausprosessi tuottaa lopputuloksena \textit{mutaatiopistemäärän} (\textit{mutation adequacy score}), jonka avulla voidaan arvioida ohjelmiston testien laadukkuutta ja kykyä havaita lähdekoodissa olevia vikoja. 


\textbf{Johdanto jäsennelty asianmukaisesti (kenelle, miksi, millaisessa ympäristössä; ratkaisun lähestymistapa; tutkimuskysymys, tulokset ja impakti),
pituus 1,5 - 2 s.}

%\textbf{\textit{MIKÄ ON MINUN TUTKIMUSKYSYMYKSENI?}}


\section{Testaus oliojärjestelmissä [2s]}

Olioperustaisen ohjelmoinnin kehityksen myötä klassisia ohjelmistojen testausmenetelmiä on sopeutettu mahdollistamaan \textit{oliojärjestelmien} (\textit{object oriented systems}) kattava ja laadukas testaaminen. Vaikka olioperustainen ohjelmointi ratkaisee joitakin proseduraalisen ohjelmoinnin suunnittelu- ja toteutusongelmia, olio-ohjelmoinnin mukana tulevat uudet haasteet vaativat uusien testaus- ja analysointimenetelmien kehittämistä. 

\subsection{Testauksen rooli oliojärjestelmissä}

Testausta käytetään ohjelmistokehityksessä ohjelmiston laadun varmistamiseen ja auttamaan virheiden havaitsemisessa jo kehitysvaiheen aikana. Ohjelmistojen testaamisen ensisijainen tavoite on siis paljastaa virheitä, joiden havaitseminen muiden laadunvarmistusmenetelmien avulla olisi työlästä tai mahdotonta~\cite[s. 59]{Binder:1999}. Testauksen avulla pyritään lisäksi varmistamaan, että ohjelma toimii sille asetettujen vaatimusten mukaisesti. 

Olio-ohjelmoinnissa testaukseen tuovat haasteita olio-ohjelmien erityispiirteet, joita ovat muun muuassa kapselointi, perintä, dynaaminen sidonta ja polymorfismi~\cite[s. 86]{Mariani:Pezze:2008}. \textbf{Lisää tähän? Binderin kirjasta löyty tietoa.}


\subsection{Testauksen tasot}

Ohjelmistoja voidaan testata usealla tasolla. Tasoja ovat yksikkö-, integraatio- ja järjestelmätasot ja ne muodostuvat joukosta ohjelman komponentteja, joita tason testit testaavat~\cite[s. 45]{Binder:1999}. Komponentteja ovat olio-ohjelmissa esimerkiksi yksittäiset metodit ja luokat, ohjelman luokkien väliset rajapinnat tai jo valmis ohjelmisto.

Alimmalla testauksen tasolla \textit{yksikkötestauksessa} (\textit{unit testing})~\cite[s. 45]{Binder:1999} testataan ohjelman pienimpiä suoritettavissa olevia komponentteja. Olio-ohjelmissa komponentteja ovat yksittäiset metodit ja oliot. \textbf{Lisää tietoa yksikkötestauksesta.}

Yksikkötestauksesta seuraava taso ylöspäin on \textit{integraatiotestaus} (\textit{integration testing})~\cite[s. 45]{Binder:1999}, jossa tarkastellaan järjestelmän tai sen osien yhteistoimintaa. Integraatiotestauksessa testataan siis järjestelmän osien välisiä rajapintoja ja osien keskinäistä kommunikointia. Olio-ohjelmissa luokkien muodostuminen perinnän avulla ja luokkien koostuminen toisten luokkien olioista aiheuttaa, että integraatiotestaukselle on olio-ohjelmoinnissa tarvetta jo ohjelmoinnin alkuvaiheessa. \textbf{Lisää ehkä vähän myös tästä.}

Valmista integroitua sovellusta testataan \textit{järjestelmätestauksen} (\textit{system testing})~\cite[s. 45]{Binder:1999} avulla. Tällä testauksen tasolla keskitytään vain valmiissa sovelluksessa esiintyvien piirteiden testaamiseen. Testauksen kohteena voi olla esimerkiksi sovelluksen toiminnallisuus, suorituskyky tai sovelluksen kestämä kuormitus~\cite[s. 45]{Binder:1999}. \textbf{Tästä jonkin verran lisää.}


\subsection{Testien suunnittelu}

Testien suunnitteluun ja kehittämiseen voidaan käyttää erilaisia menetelmiä. Testausmenetelmän avulla kuvataan näkökulmaa, josta ohjelman lähdekoodia tarkastellaan testejä kehitettäessä~\cite[s. 51]{Binder:1999}. Testausmenetelmiä ovat esimerkiksi white box - ja black box -testaus sekä niitä yhdistävä hybriditestaus. Lisäksi testien laadun parantamiseen voidaan käyttää virheisiin perustuvaa testausmenetelmää~\cite{Joku mutaatiolähde tähän.}.

Ohjelman sisäisen rakenteen eli lähdekoodin tuntemukseen perustuvaa testausmenetelmää kutsutaan \textit{white box -testaukseksi}~\cite[s. 52]{Binder:1999}. \textbf{Lisää tästä.} White box -testausta voidaan käyttää esimerkiksi yksikkötestauksessa apuna testien suunnittelussa, sillä lähdekoodin tuntemus auttaa kehittämään testejä yksittäisille metodeille ja olioille~\cite{JOKU??}.

\textit{Black box - testaukseksi} eli \textit{funktionaaliseksi testaukseksi}~\cite[s. 52]{Binder:1999} kutsutussa testausmenetelmässä testejä suunnitellaan analysoimalla ohjelmiston ulkoista toiminnallisuutta. \textbf{Lisää tuosta.} Koska valmiin sovelluksen piirteitä testatessa tutkitaan myös sovelluksen ulkoista toiminnallisuutta, on black box -testauksesta apua esimerkiksi suunniteltaessa testejä järjestelmätestaukseen~\cite{Mistä lie?}.

\textit{Gray box -} eli \textit{hybriditestauksessa}~\cite[s. 52]{Binder:1999} yhdistetään white box - ja black box -testausmenetelmien piirteitä. Näinollen sekä white box - että black box -testausmenetelmää voidaan käyttää testien suunnittelussa useilla testauksen tasoilla joko erikseen tai molempien piirteitä yhdistellen.

Testausmenetelmää, jossa ohjelman lähdekoodiin lisätään virheitä, kutsutaan \textit{virheperustaiseksi testausmenetelmäksi} (\textit{fault-based testing})~\cite[s. 52]{Binder:1999}. Esimerkkinä virheperustaisesta testausmenetelmästä on mutaatiotestaus, jonka avulla tutkitaan testien kykyä havaita ohjelmiston virheitä~\cite[s. X]{Joku mutaatiolähde}.


\subsection{Testauksen rajoitukset}

Yksi testaukseen liittyvistä rajoituksista on, että testauksen avulla ei voi aina todeta ohjelmiston oikeellisuutta. Jotta oikeellisuus voidaan todistaa, vaaditaan, että ohjelman oikea toiminta testataan kaikilla mahdollisilla syötteillä ja niiden kombinaatioilla. Ohjelman oikeellisuuden todistaminen vastaa siis ohjelman kattavaa testaamista  \textbf{\textit{exhaustive testing}}. Kattava testaaminen on kuitenkin käytännössä usein mahdotonta toteuttaa muille kuin triviaaleille ohjelmille~\cite[s. 58]{Binder:1999} eli se on \textit{\textbf{intractable?? ongelma}}. Oikellisuuden todistamiseen liittyen Edsger Dijkstra totesikin: \textit{''Program testing can be used to show the presence of bugs, but never to show their absence!''}~\cite[s. 6]{Dahl:Dijkstra:Hoare:1972}.

Suoritettujen testien tulosten tulkintaan liittyy myös rajoituksia ja epävarmuutta. Epävarmuus ilmenee, kun testien lopputuloksia varten ei ole olemassa luotettavia odotettuja tuloksia  vertailukohdaksi \textit{\textbf{expected results vs actual results}}. Tällöin testauksesta saatujen todellisten tulosten tulkinta ja arviointi on epävarmaa~\cite[s. 58]{Binder:1999} eli todellisista tuloksista ei voi luotettavasti päätellä, menivätkö testit läpi vai eivät. 

Epävarmuutta liittyy myös testauksen kohteen \textit{\textbf{System under test SUT}} toiminnalle asetettuihin vaatimuksiin. Vaatimusten todentaminen testauksen avulla ei ole mahdollista, joten vaatimuksia tulee käyttää testauksen tulosten tulkinnassa vertailukohtana \textit{\textbf{point of reference}}~\cite[s. 58]{Binder:1999}. Jos virheellisiä tai puutteellisia vaatimuksia käytetään testejä tehdessä, voi siitä seurata harhaanjohtavia testejä. Toisaalta, vaikka testausmenetelmänä olisi white box -testaus eli lähdekoodin tuntemusta käytettäisiin hyväksi testejä kehitetäessä, white box -testausmenetelmän avulla ei voi paljastaa lähdekoodista puuttuvia osia \textit{\textbf{omission??}}~\cite[s. 58]{Binder:1999}. Tämä on seurausta siitä, että koodia, jota ei ole olemassa, ei voi myöskään testata.

\section{Mutaatiotestaus oliojärjestelmissä [3s]}

\textbf{Haasteisiin voidaan vastata mutaatiotestauksen avulla.}

Vaikka testauksen avulla ei voikaan varmistua ohjelmiston oikeellisuudesta, voidaan testausta käyttää välineenä ohjelmiston laadun parantamisessa. Testaukseen sisältyvien rajoitusten lisäksi siihen liittyy myös epävarmuutta käytettävän testausjärjestelmän oikeellisuudesta ja oikeellisuuden varmistamisesta~\cite[s. 209]{Manna:Waldinger:1978}. Tämä herättää kysymyksen siitä, kuka voi ''valvoa valvojia'' eli kuinka varmistetaan ohjelmiston testien laadukkuus.

\subsection{Mutaatiotestauksen yleisesittely}

Yksi strategia testien laadun varmistamiseen on \textit{mutaatiotestaus}. Mutaatiotestauksessa ohjelmiston alkuperäistä lähdekoodia käsitellään \textit{mutaatio-operaattoreilla} (\textit{mutation operators}), jotka muuntavat koodia muodostaen siitä virheellisiä versioita\textbf{Opettaja halusi lähteen tähän mielummin kuin käsitteen perään}. Näitä virheellisiä ohjelmakoodin versioita kutsutaan \textit{mutanteiksi}~\cite[s. 869]{Ma:Harrold:Kwon:2006}. 

Mutanttien generoinnin jälkeen ohjelmiston alkuperäiset testit suoritetaan jokaisen mutantin kohdalla. Tavoitteena on, että testien avulla havaitaan lähdekoodiin tehdyt muutokset. Jos alkuperäiset testit eivät mene läpi, se tarkoittaa, että mutantti on tapettu eli lähdekoodiin tehdyt muutokset on havaittu~\cite[s. 9]{Kim:Clark:McDermid:2000}. 

Mutaatiotestausprosessi tuottaa lopputuloksena \textit{mutaatiopistemäärän} (\textit{mutation adequacy score}), jonka avulla voidaan arvioida ohjelmiston testien laadukkuutta ja kykyä havaita lähdekoodissa olevia vikoja\textbf{Lähde taas tähän lauseen loppuun mielummin kuin tuonne käsitteen perään}. 

Mutaatiotestaus on virheperustainen testausmenetelmä, jonka periaatteena on ohjelmoijien tekemien ohjelmointivirheiden simulointi~\cite[s. 649]{Jia:Harman:2011}. Tavoitteena mutaatiotestauksessa on tutkia, ovatko ohjelmistoa varten tehdyt testit laadukkaita ja havaitaanko niillä kattavasti ohjelmistossa mahdollisesti esiintyvät virheet ja ongelmat. 


\subsection{Mutaatiotestauksen piirteet oliojärjestelmissä (erot muihin järjestelmiin?)}

\textbf{Erityisesti juuri nuo mutaatio-operaattorit ja minkälaisia erilaisia niitä on. Esittelyä tulee mutaatio-operaattoreista.}


\section{Mutaatiotestauksen haasteet [3s]}

\textbf{Misi mutaatiotestaus ei ole päätynyt suureen suosioon/käyttöön? Valitaan muutama haaste ja esitellään niitä? Kerrotaan että myös muita haasteita, mutta ei esitellä niitä niin tarkasti?}

\textbf{\textit{Haasteet sisältäen esimerkin, miten se yksittäinen haaste on yritetty ratkaista (eli joko ratkaisu tai ratkaisuehdotus)??}}

\subsection{Ekvivalentit mutantit}

\textbf{Ratkaisematon ongelma. Pitäisi avata ja kehittää esimerkki.}

\subsection{Tehokkuusongelmat}

\textbf{Laitteisto, testien määrä, ihmisten aika tulee vastaan. Esimerkkejä.}

\subsection{Muita ongelmia?}

\textbf{Varmasti on muita, mutta kuinka paljon niitä mahtuu tähän esitelmään?}

\subsection{Käytännöntoteutus (miten käytännössä on yritetty kiertää haasteet)}

\textbf{Onko tämä relevantti aihealue käsitellä?}

\section{Miten mutaatiotestaus auttaa testien laadun parantamisessa?}


\subsection{Miksi mutaatiotestausta tulisi käyttää, vaikka se on raskasta ja työlästä?}

\subsection{Miltä tulevaisuus näyttää, tuleeko käyttö lisääntymään vai jääkö mutaatiotestaus unohduksiin/pienen piirin harrastukseksi?}


\section{Yhteenveto [1s]}

\textbf{Conclusion.}

 






 


%Tällä (yksikkö)testauksen tasolla testien suunnittelussa voidaan käyttää ohjelman sisäiseen rakenteeseen perustuvaa suunnittelumallia, jota kutsutaan \textit{white box -testaukseksi} (\textit{white box testing}). Tässä mallissa ohjelmiston lähdekoodia käytetään apuna testien valmistamisessa~\cite[s. 52]{Binder:1999}.

%Tämän tason testien suunnittelustrategiana voidaan käyttää ohjelmiston ulkoisen toiminnallisuuden analysointia. Tällaista testausstrategiaa kutsutaan \textit{black box -testaukseksi} (\textit{black box testing}) tai \textit{funktionaaliseksi testaukseksi} (\textit{functional testing})~\cite[s. 52]{Binder:1999}. Kuitenkin sekä white box - että black box -testausta voidaan käyttää kaikilla testauksen tasoilla testien suunnittelun apuna. 


%\section{Miten saa evaluointia/analyysiä? Mihin lukuihin se sopii?}

%\textbf{Mihin kohtiin evaluointi ja analyysi sopivat, jotta tästä ei tule mutaatiotestauksen/testauksen oppikirjaa? Miten sitä tehdään? Vertailemalla, ennustamalla tulevaa, jotakin muuta?}

%Evaluointi ja analyysi ovat jossain määrin vähemmän sisältäviä kuin voisi kuvitella. Analyysi esim. on pääasiassa vaikka asioiden vertailua (koska omaa tutkimusta ei tehdä). Evaluointi taas on esimerkiksi tiedon rajaamista ja esittämistä yhdisteltynä toisiin tietoihin. Ts. olet joutunut itse keräämään tiedon ja rajaamaan sen sopivaksi joukoksi infoa tätä tutkielmaa varten. Siitä muodostuu evaluointi osa.


% --- References ---
%
% bibtex is used to generate the bibliography. The babplain style
% will generate numeric references (e.g. [1]) appropriate for theoretical
% computer science. If you need alphanumeric references (e.g [Tur90]), use
%
\newpage
\bibliographystyle{babalpha-lf}
%
% instead.

%\bibliographystyle{babplain-lf}
\bibliography{references-fi}


% --- Appendices ---

% uncomment the following

% \newpage
% \appendix
% 
% \section{Esimerkkiliite}

\end{document}
